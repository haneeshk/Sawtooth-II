\begin{figure}[H]
\centering
\includegraphics[width=\textwidth]{Figures/Figure5_V2.pdf}%V3?
\caption{Correlation between the sawtooth pattern in the force-displacement response and discontinuous jumps in arc length. (\textsf{A}) A schematic of a spicule's deformed configuration in the simply-supported tests showing the arc length $S$. (\textsf{B}) Magnified view of (\textsf{A}) showing the points $(^{i}x_1,^{i}x_2)_{i=1\ldots n}$ identified along the spicule's longitudinal axis and the continuous representation of the longitudinal axis $f:[0,L]\rightarrow \mathbb{R}$. (\textsf{C}) A graph of the function $f$ for a representative spicule computed for every fifth stage displacement increment. (\textsf{D}) The force, $F$, (left axis) as a function of stage displacement, $w_s$, for a representative spicule compared to the total change in arc length $\Delta S$ (right axis). The vertical, dashed lines indicate the $w_s$ values at which we identified discontinuities (numbered 1--6) in the $F$-$w_s$ response. (\textsf{E}) A zoomed in view of the plot region within the red rectangle in (\textsf{D}). (\textsf{F}) A zoomed in view of the plot region within the red rectangle in (\textsf{E}) highlighting the discontinuous changes in $\Delta S$ at locations 1--3. (\textsf{G}) A histogram of the total change in arc length at the initiation of failure $\Delta S^*$ for the 22 spicules exhibiting sawtooth patterns in their $F$-$w_0$ responses.}
\label{fig:slip}
\end{figure}
